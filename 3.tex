\documentclass{article}
\usepackage{mathtools} 
\usepackage{fontspec}
\usepackage[UTF8]{ctex}
\usepackage{amsthm}
\usepackage{mdframed}
\usepackage{xcolor}
\usepackage{amssymb}
\usepackage{amsmath}

\newmdtheoremenv[
  backgroundcolor=gray!10,
  linewidth=0pt,
  innerleftmargin=10pt,
  innerrightmargin=10pt,
  innertopmargin=10pt,
  innerbottommargin=10pt
]{zgraytheorem}{}
% 定义说明环境样式
\newtheoremstyle{mystyle}% 说明环境样式的名称
  {1em}% 上方间距
  {1em}% 下方间距
  {\normalfont}% 说明内容的字体样式
  {}% 缩进量
  {\bfseries}% 说明标记的字体样式
  {.}% 说明标记和说明内容之间的标点
  {1em}% 说明标记后的水平空间
  {}% 说明标记后的垂直空间
% 使用新定义的样式创建说明环境
\theoremstyle{mystyle}
\newtheorem*{zremark}{说明}

% 定义证明环境样式
\newtheoremstyle{zproofstyle}
  {0.5em}
  {0.5em}
  {\itshape}
  {}
  {\bfseries}
  {.}
  {\newline}
  {}
\theoremstyle{zproofstyle}
\newtheorem*{zproof}{证明}
\newcommand{\zsub}{\mathbin{\rule{1em}{0.5em}}}
\begin{document}
\title{第三章 习题}
\maketitle

\section*{习题 3.18}
\begin{zproof}
    不妨设$\phi$其中出现的命题都在$q_0,...,q_n$之中。
    设$s$为任意代入,$\sigma$为任意真值指派,
    可以定义出一个真值指派$\sigma^\prime$,
    使得$s(q_0)^\sigma = q_0^{\sigma^\prime},...,s(q_n)^\sigma = q_n^{\sigma^\prime}$,
    因为$\phi$是重言式,所以$\phi$在任意真值指派下的值都是T,所以${\sigma^\prime} \models \phi$,
    由命题90可知$\sigma \models \phi(s)$。
\end{zproof}
\end{document}